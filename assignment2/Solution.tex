%%This is a very basic article template.
%%There is just one section and two subsections.
\documentclass[a4paper, 11pt]{article}
\usepackage[pdftex]{graphicx}
\usepackage{parskip}
\usepackage{hyperref}
\usepackage[all]{hypcap}
\usepackage{amsmath}
\usepackage{amsfonts}
\usepackage{enumitem}
\title{Assignment 2 \\ Math 205 Linear Algebra}

\author{
  Muhammad Shahrom Ali\\ ma03559
  \and
  Syed Ammar Mahdi\\  sm03691
}

\newcommand{\mat}[1]{\boldsymbol { \mathsf{#1}} }

\begin{document}
\setlength{\parskip}{10pt} % 1ex plus 0.5ex minus 0.2ex}
\setlength{\parindent}{0pt}
\DeclareGraphicsExtensions{.pdf,.png,.gif,.jpg}
\maketitle




\section{Answers}
%Remove the question statements when question complete


\begin{enumerate} 

\item We have already indirectly talked about the notions of \textbf{span, independent vectors, basis} and \textbf{dimension}, but now you have to put the intuition we built in proper mathematical terms. 
The \textbf{span} has already gotten a crucial amount of attention, so we just add the formal definition here:\\


For $\vec v_1,..., \vec v_k$ vectors in $\mathbb{R}^n$, the \textit{span} of $\vec v_1,..., \vec v_k$ is the set of all linear combinations of those vectors, i.e.
$$\text{Span} \{\vec v_1,..., \vec v_k \} = \{ x_1 \vec v_1 + ... + x_k \vec v_k | x_1,...x_k \in \mathbb{R}\}$$
We also say that $\text{Span} \{\vec v_1,...,\vec v_k \} $ is the subset \textit{spanned by} or \textit{generated by} the vectors $\vec v_1,..., \vec v_k$. \\
Conversely we say that a set of vectors \textit{spans} a space if their linear combination fills the space, i.e. each vector in the space can be represented by a linear combination of those vectors. \\

In order to understand the other notions read the following chapters \url{https://textbooks.math.gatech.edu/ila/linear-independence.html} and \url{https://textbooks.math.gatech.edu/ila/dimension.html} and  summarize those three other concepts before answering the following questions (including explanation): 
\begin{enumerate}[label=\alph*)]
\item Are the vectors $\Bigg\{ \begin{bmatrix} 1 \\ 4 \\ 5 \\ 2 \end{bmatrix}, \begin{bmatrix} 3 \\ 0 \\ -1 \\ 4 \end{bmatrix}, \begin{bmatrix} 1 \\ 1 \\ -2 \\ 1
\end{bmatrix}, \begin{bmatrix} -1 4\\ 13 \\ 7 \\ -19
\end{bmatrix}\Bigg\}$ linearly dependent?
\item Can the columns of a wide matrix, $n>m$ be linearly independent?
\item Can the columns of a tall matrix, $n<m$ be linearly independent?
\item  If the columns of $\mat A$ are linearly independent, how many solutions are there to the
system $\mat A \vec x = 0$?
\item  To determine whether a set of $n$ vectors from $\mathbb{R}^n$ is independent, we can form a matrix $\mat A$ whose columns are the vectors in the set and then put that matrix in reduced row
echelon form. If the vectors are linearly independent, what will we see in the reduced
row echelon form?
\item If your vector space $V$ has dimension $m$, do any $m$ vectors in $V$ form a basis of V? 
\end{enumerate}



\item Suppose $\mat A = \begin{bmatrix} 
    3 & 6 & 9 \\ 
    6 & 9 & 12 \\
    9 & 12 & 15 \end{bmatrix}$
    \begin{itemize}
      \item Find $N(\mat A)$
      \item $\mat B = \begin{bmatrix} 1 & -1 \\ -1 & 1 \end{bmatrix}$. Find $N(\mat B)$ and $N(\mat B^2)$
      \item What can you say about the relationship between $N(\mat C)$ and $N(\mat C^2)$. Assume for this part that $\mat C$ can be any arbitrary square matrix
    \end{itemize}
    
\item A linear system $\mat A \vec x = \vec b$ has special solution of the form
\begin{align}
 \vec x_n = x_2 \begin{bmatrix} 1 \\ 1 \\ 0 \\ 0 \end{bmatrix}
 \end{align} 
\begin{enumerate}[label=(\alph*)]
\item What are the basis for $N(\mat A)$
\item What is the dimension of $N(\mat A)$
\item Write the reduced row echelon form for $\mat A$
\item Describe intuitively the geometry of the solution space
\end{enumerate}

\item Determine with reason which of the following are subspaces of $3 \times 3$ matrix $\mat M$
\begin{enumerate}[label=(\alph*)]
\item all $3 \times 3$ matrices $\mat A$ such that $\mat A^T = -\mat A$
\item all $3 \times 3$ matrices such that the linear system $\mat A \vec x = \vec 0$ has only trivial solution
\end{enumerate}



\item For which right sides are these systems solvable? Give your reasons
\begin{enumerate}
\item
\[ 
\left[ \begin{array}{ccc}
5  &   7 &  5\\
10  &   14 &  15\\
20 & 28 & 25
\end{array} \right]
%
\left[ \begin{array}{c}
 x_1\\
 x_2\\
x_3 
\end{array} \right]
%
= \left[ \begin{array}{c}
 b_1\\
 b_2\\
 b_3 
\end{array} \right]
\]

For all $\vec b$ in the $C(A)$, there exists a solution for the system.Applying row operations on $\mat A$: 

\begin{center}
$-2R_1 + R_2$

$-4R_1 + R_3$

$-R_3 + R_1$

$-R_3 + R_2$

$\frac{1}{5} R_1$

$\frac{1}{5} R_3$

$ R_2 \leftrightarrow R_3$
\end{center}


\[ 
\left[ \begin{array}{ccc}
1  & \frac{7}{5} &  0\\
0  &   0 &  5\\
0 & 0 & 0
\end{array} \right]
\]

$\vec a_1$ and $\vec a_3$ are linearly independent and their span is the $C(A)$ hence 

$\forall \vec b \in C(A)$  the system is solvable.


\item
\[ 
\left[ \begin{array}{cc}
1  &   4 \\
2  &  9 \\
-1 & -4 
\end{array} \right]
%
\left[ \begin{array}{c}
 x_1\\
 x_2\\
\end{array} \right]
%
= \left[ \begin{array}{c}
 b_1\\
 b_2\\
 b_3 
\end{array} \right]
\]
\item 
\[ 
\left[ \begin{array}{ccc}
1  & 1 & 1\\
0  & 1 & 1\\
0 & 0 & 1
\end{array} \right]
%
\left[ \begin{array}{c}
 x_1\\
 x_2\\
x_3
\end{array} \right]
%
= \left[ \begin{array}{c}
 b_1\\
 b_2\\
 b_3 
\end{array} \right]
\]
\item
\[
 \left[ \begin{array}{ccc}
1  & 1 & 1\\
0  & 1 & 1\\
0 & 0 & 0
\end{array} \right]
%
\left[ \begin{array}{c}
 x_1\\
 x_2\\
x_3
\end{array} \right]
%
= \left[ \begin{array}{c}
 b_1\\
 b_2\\
 b_3 
\end{array} \right]
\]

\end{enumerate}

\item Given that $\mat A$ is an arbitrary $4 \times 3$ matrix, if we add an extra column $\vec a_4$ to a matrix $\mat A$. then the column space gets larger unless $\_ \_ \_\_ \_ \_\_ \_ \_\_ \_ \_\_ \_ \_\_ \_ \_$. Give an example where the column space gets larger and an example where it doesn't. 

What should be the condition on $\vec b$ for $\mat A \vec x = \vec b$ to have a solution if the $C(\mat A)$ doesn't get larger?  

\item We begin by making a matrix of all of these vectors. 

\begin{center}

$
\begin{bmatrix}
\vec v_1 & \vec v_2 & \vec v_3 & \vec v_4 & \vec v_5 & \vec v_6  
\end{bmatrix}
$

$
\begin{bmatrix}
1 & 1 & 1 & 0 & 0 & 0 \\
-1 & 0 & 0 & 1 & 1 & 0\\
0 & -1 & 0 & -1 & 0 & 1\\ 
0 & 0 & -1 & 0 & 1 & -1 
\end{bmatrix}
$

\end{center}

We then perform row operations on this matrix:
\begin{center}
$R_1 + R_2$

$R_2 + R_3$

$R_3 + R_4$

$\frac{1}{2} R_4$
\end{center}

The Matrix then becomes 

\begin{center}
$
\begin{bmatrix}
1 & 1 & 1 & 0 & 0 & 0 \\
0 & 1 & 1 & 1 & 1 & 0\\
0 & 0 & 1 & 0 & 1 & 1\\ 
0 & 0 & 0 & 0 & 1 & 0 
\end{bmatrix}
$
\end{center}

Out of these we can see $\vec v_1, \vec v_4, \vec v_6,$ and $\vec v_5$ to be linearly independent.

\textbf{So we have atmost 4 linearly independent vectors.}


\item Find the bases for the $C(\mat \cdot)$ and $N(\mat \cdot)$ associated with $\mat A$ and $\mat B$:
\[ A = \left[ \begin{array}{ccc}
1&2&4\\
2&4&8
\end{array} \right]
\hspace{10mm}, B = \left[ \begin{array}{ccc}
1&2&4\\
2&5&8
\end{array} \right]\]

\item If $\mat V$ is the subspace spanned by (1, 1, 1) and (2, 1, 0), find a matrix $\mat A$ that has $\mat V$ as its column space and a matrix $\mat B$ that has $\mat V$ as its nullspace.

\item Find the complete solution for the following equations and describe the solution space:
\begin{equation} \label{eq1}
\begin{split}
x + 3y + 3z  = 0\\
2x + 6y + 9z = 0\\
-x - 3y +3z  = 0.
\end{split}
\end{equation}
\end{enumerate}
\end{document}